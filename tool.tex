\section{CodeInk: an algorithm animation tool}


% CodeInk:
%- Objective: enable teachers and students to explain algorithms by directly
% manipulating data structures.
%- Scope: traces and visual problems
%- Usage model through example
%- Use cases

%The primary contribution of this paper is a direct manipulation language
%for explaining algorithms. To realize and evaluate this language, we
%implemented a prototype tool called CodeInk, which serves as our
%%details of our language.

CodeInk (\fig{fig:codeink-intro}) is a Web-based algorithm animation tool that
enables instructors and students to explain algorithms by directly manipulating
data structures. It implements the direct manipulation language we present in the
next section, enabling us to evaluate the language's usability and benefits.
CodeInk's design is motivated by the common convention of teaching and learning
algorithms with worked examples~\cite{Sweller1985} -- step-by-step
demonstrations of how to solve a problem on concrete examples.
\fig{fig:6006-insertion}, for instance, shows how instructors often explain
insertion sort in an introductory algorithms class: they trace the algorithm's
steps on a concrete list rather than, say, analyze abstract pseudocode.
Due to the limitations of the blackboard, the instructor is forced to show
element swaps by redrawing the list in each stage of the sort.

CodeInk eliminates the tedium of drawing such storyboard diagrams, making it possible to
manipulate the data structure directly. Its basic usage model is:

\begin{enumerate}

\item Set up a concrete example of a data structure (list, binary tree,
graph) by dragging constituent objects onto the canvas and
initializing their values.

\item Directly manipulate the data structure using gestures to
demonstrate an algorithm's trace on that data.

\item Every interaction adds a step to the trace, translated into Python
code (\fig{fig:codeink-intro}c). Click on any step to jump to the
corresponding point in the trace, or play back the trace using the
controls in \fig{fig:codeink-intro}b.

\item Share the explanation online by generating a URL, embedding into a
Web page, or exporting as a video.

\end{enumerate}



%% CodeInk, VL/HCC 2014 Paper Submission
%% 2014/02/01
%% (requires IEEEtran.cls version 1.7 or later) with an IEEE conference paper.


\documentclass[conference]{IEEEtran}

% Some very useful LaTeX packages include:
% (uncomment the ones you want to load)

% *** CITATION PACKAGES ***
%
%\usepackage{cite}
% cite.sty was written by Donald Arseneau
% V1.6 and later of IEEEtran pre-defines the format of the cite.sty package
% \cite{} output to follow that of IEEE. Loading the cite package will
% result in citation numbers being automatically sorted and properly
% "compressed/ranged". e.g., [1], [9], [2], [7], [5], [6] without using
% cite.sty will become [1], [2], [5]--[7], [9] using cite.sty. cite.sty's
% \cite will automatically add leading space, if needed. Use cite.sty's
% noadjust option (cite.sty V3.8 and later) if you want to turn this off.
% cite.sty is already installed on most LaTeX systems. Be sure and use
% version 4.0 (2003-05-27) and later if using hyperref.sty. cite.sty does
% not currently provide for hyperlinked citations.
% The latest version can be obtained at:
% http://www.ctan.org/tex-archive/macros/latex/contrib/cite/
% The documentation is contained in the cite.sty file itself.

\usepackage{color}

\hyphenation{geo-me-tric}
\hyphenation{ma-ni-pu-la-tion}

% *** GRAPHICS RELATED PACKAGES ***
%
\ifCLASSINFOpdf
  \usepackage[pdftex]{graphicx}
  % declare the path(s) where your graphic files are
  \graphicspath{{img/}}
  % and their extensions so you won't have to specify these with
  % every instance of \includegraphics
  \DeclareGraphicsExtensions{.pdf,.jpeg,.png}
\else
  % or other class option (dvipsone, dvipdf, if not using dvips). graphicx
  % will default to the driver specified in the system graphics.cfg if no
  % driver is specified.
  % \usepackage[dvips]{graphicx}
  % declare the path(s) where your graphic files are
  % \graphicspath{{../eps/}}
  % and their extensions so you won't have to specify these with
  % every instance of \includegraphics
  % \DeclareGraphicsExtensions{.eps}
\fi
% graphicx was written by David Carlisle and Sebastian Rahtz. It is
% required if you want graphics, photos, etc. graphicx.sty is already
% installed on most LaTeX systems. The latest version and documentation can
% be obtained at: 
% http://www.ctan.org/tex-archive/macros/latex/required/graphics/
% Another good source of documentation is "Using Imported Graphics in
% LaTeX2e" by Keith Reckdahl which can be found as epslatex.ps or
% epslatex.pdf at: http://www.ctan.org/tex-archive/info/
%
% latex, and pdflatex in dvi mode, support graphics in encapsulated
% postscript (.eps) format. pdflatex in pdf mode supports graphics
% in .pdf, .jpeg, .png and .mps (metapost) formats. Users should ensure
% that all non-photo figures use a vector format (.eps, .pdf, .mps) and
% not a bitmapped formats (.jpeg, .png). IEEE frowns on bitmapped formats
% which can result in "jaggedy"/blurry rendering of lines and letters as
% well as large increases in file sizes.
%
% You can find documentation about the pdfTeX application at:
% http://www.tug.org/applications/pdftex

% *** SPECIALIZED LIST PACKAGES ***
%
%\usepackage{algorithmic}
% algorithmic.sty was written by Peter Williams and Rogerio Brito.
% This package provides an algorithmic environment fo describing algorithms.
% You can use the algorithmic environment in-text or within a figure
% environment to provide for a floating algorithm. Do NOT use the algorithm
% floating environment provided by algorithm.sty (by the same authors) or
% algorithm2e.sty (by Christophe Fiorio) as IEEE does not use dedicated
% algorithm float types and packages that provide these will not provide
% correct IEEE style captions. The latest version and documentation of
% algorithmic.sty can be obtained at:
% http://www.ctan.org/tex-archive/macros/latex/contrib/algorithms/
% There is also a support site at:
% http://algorithms.berlios.de/index.html
% Also of interest may be the (relatively newer and more customizable)
% algorithmicx.sty package by Szasz Janos:
% http://www.ctan.org/tex-archive/macros/latex/contrib/algorithmicx/


% *** ALIGNMENT PACKAGES ***
%
\usepackage{array}
% Frank Mittelbach's and David Carlisle's array.sty patches and improves
% the standard LaTeX2e array and tabular environments to provide better
% appearance and additional user controls. As the default LaTeX2e table
% generation code is lacking to the point of almost being broken with
% respect to the quality of the end results, all users are strongly
% advised to use an enhanced (at the very least that provided by array.sty)
% set of table tools. array.sty is already installed on most systems. The
% latest version and documentation can be obtained at:
% http://www.ctan.org/tex-archive/macros/latex/required/tools/

% *** PDF, URL AND HYPERLINK PACKAGES ***
%
%\usepackage{url}
% url.sty was written by Donald Arseneau. It provides better support for
% handling and breaking URLs. url.sty is already installed on most LaTeX
% systems. The latest version can be obtained at:
% http://www.ctan.org/tex-archive/macros/latex/contrib/misc/
% Read the url.sty source comments for usage information. Basically,
% \url{my_url_here}.

% *** Do not adjust lengths that control margins, column widths, etc. ***
% *** Do not use packages that alter fonts (such as pslatex).         ***
% There should be no need to do such things with IEEEtran.cls V1.6 and later.
% (Unless specifically asked to do so by the journal or conference you plan
% to submit to, of course. )


% correct bad hyphenation here
\hyphenation{op-tical net-works semi-conduc-tor}


% comment-out to make sure there are no more todos
\newcommand{\todo}[1]{\textcolor{red}{\textbf{[#1]}}}


% Consistent cross referencing
\def\sec#1{Section~\ref{#1}}
\def\fig#1{Figure~\ref{#1}}
\def\tab#1{Table~\ref{#1}}
\def\eqn#1{Equation~\ref{#1}}



%\usepackage{setspace}
%\doublespace
\begin{document}
%
% paper title
% can use linebreaks \\ within to get better formatting as desired
%\title{CodeInk: A Direct Manipulation Language \\ for Explaining Algorithms}
\title{CodeInk: Explaining Algorithms \\ with Direct Manipulation}


% author names and affiliations
% use a multiple column layout for up to three different
% affiliations
\author{\IEEEauthorblockN{Jeremy Scott}
\IEEEauthorblockA{MIT CSAIL\\
Cambridge, MA 02139\\
Email: jscott@csail.mit.edu}
\and
\IEEEauthorblockN{Philip J.\ Guo}
\IEEEauthorblockA{MIT CSAIL / University of Rochester \\
Cambridge, MA 02139\\
Email: pg@cs.rochester.edu}
\and
\IEEEauthorblockN{Randall Davis}
\IEEEauthorblockA{MIT CSAIL\\
Cambridge, MA 02139\\
Email: davis@csail.mit.edu}}

% conference papers do not typically use \thanks and this command
% is locked out in conference mode. If really needed, such as for
% the acknowledgment of grants, issue a \IEEEoverridecommandlockouts
% after \documentclass

% make the title area
\maketitle

\begin{abstract}

% 1-2 sentences: motivation
Tracing algorithms on concrete examples is fundamental to how instructors explain
and how students learn an algorithm. On both blackboards and paper 
tracing is typically done by drawing an example data structure and a storyboard showing how it's transformed by
the algorithm. But hand drawing is tedious and limiting
because data structures must be erased and redrawn, and the
step-by-step trace of the algorithm cannot be easily shared or discussed.
% 1 sentence: hypothesis / general idea
We present CodeInk, a Web-based system for tracing algorithms that enables the
user to directly manipulate data structures.
%%% said this record/share stuff below
% 1 sentence: how things work / how the contributions were achieved
The system's direct manipulation gesture set enables the user to demonstrate the algorithm as a
process, instead of drawing static snapshots of its effects. 
All gestures are translated into Python and recorded,
enabling navigation through the trace. Steps can also be shared or analyzed as a basis for
feedback to students.
% 1-2 sentences: evaluation results 1 sentence: this paper describes,
% contributions
In a user study we found that students learned equally well from
CodeInk-produced traces as they did from lecture videos, and that they
were then able to trace the algorithm for themselves in CodeInk.

\end{abstract}


% no keywords

% For peer review papers, you can put extra information on the cover
% page as needed:
% \ifCLASSOPTIONpeerreview
% \begin{center} \bfseries EDICS Category: 3-BBND \end{center}
% \fi
%
% For peerreview papers, this IEEEtran command inserts a page break and
% creates the second title. It will be ignored for other modes.
\IEEEpeerreviewmaketitle

\section{Introduction}

%Tracing algorithms is ubiquitous, fundamental
Tracing algorithms on concrete examples is fundamental to how teachers and students in
computer science explain and learn, respectively. On the blackboard, teachers explain an
algorithm for the first time by drawing an example data structure and carrying
out the algorithm's operation on the example. Similarly, before
attempting problem sets and coding assignments,
students often review algorithms by reading pseudocode and
tracing through multiple examples on paper~\cite{Vainio2007}.
% mentally tracing the algorithm's behavior

% Tedious, limiting no manipulation affordances no persistent, structured
% recording
This process of tracing algorithms on blackboards and paper can be both tedious
and limiting. First, blackboards and paper do not afford manipulation of the
drawing. To demonstrate changes, the user must erase, redraw, and storyboard the
behavior as a series of static snapshots.
Second, the drawing process is not recorded in a persistent, structured format,
so it can be hard for students and teachers to share ideas and discuss problems
unless they are co-located.
This limitation is particularly problematic in a MOOC context, where the
majority of students do not have direct access to teaching staff.

\begin{figure}

\begin{center}
%\includegraphics[width=0.55\columnwidth]{img/frontpage-6006.png}
\includegraphics[width=\columnwidth]{img/frontpage-mergesort.png}
\end{center}

\caption{CodeInk is a Web-based tool that implements a direct
manipulation gesture set for tracing algorithms on data structures for the
purpose of CS education. CodeInk allows the user to (a) compose and
manipulate data structures on a graphical canvas, (b) step forward and
backward through the recorded trace, and (c) see each step translated
into a line of Python code or an English explanation.}

%The CodeInk user interface is (a) a canvas for composing and
%manipulating data structures and (c) an interactive list of Python steps
%recorded by the tool, based on the user's interactions. The user can
%step forward and backward through the steps, or replay the entire trace
%using the playback controls (b).}

\label{fig:codeink-intro}
\end{figure}

% Code-driven visualization is for visualization, not explanation or active
% learning Also at the wrong level of abstraction
Code-driven program visualizations~\cite{Guo2013, Sorva2013} are an alternative
to manually drawing traces on blackboards or paper. Using such a tool, the user
can enter code that implements an algorithm and then watch a step-by-step
animation of its execution.
However, the implementation of an algorithm is usually at a lower level of
abstraction than the language-agnostic level at which the algorithm should be
explained and learned. Also, watching a visualization is passive learning,
whereas tracing requires the student to play the role of the computer and carry
out the algorithm's behavior.
This form of active learning has been shown to result in better learning
outcomes~\cite{Sorva2012Diss}.

We hypothesize that the ideal user interface for teaching and learning
algorithms should support exploration by tracing, and enhance the experience by
(a) affording users the ability to directly manipulate data structures, and
(b) recording the trace in a structured, persistent format that affords
dissemination, interaction and discussion.

To explore this hypotheiss, we built \emph{CodeInk}, a Web-based tool for
tracing algorithms by direct manipulation. It includes a novel direct
manipulation (DM) gesture set that enables users to transform lists, trees and
graphs rather than tediously draw before-and-after snapshots. In CodeInk, when
the user manipulates data structures, a trace is recorded as interactive program
steps, where each gesture is translated to a line of Python code or explanatory
English. The steps provide three main benefits: feedback on user interactions,
navigation through the trace, and a format that can be easily disseminated and
analyzed as a basis for discussion.

\fig{fig:codeink-intro} shows how an instructor can explain the merge sort
algorithm in CodeInk by dragging an example list onto the canvas, selecting
sublists with a rectangular selection, dragging them away to create copies, then
merging elements by dragging them into a new sorted list
(\fig{fig:codeink-intro}a).
Every interaction is interpreted as a step in Python (\fig{fig:codeink-intro}c).
The trace can then be shared with students, who can navigate through it by
single-stepping or clicking on steps (\fig{fig:codeink-intro}b), and then trace
the algorithm for themselves on a new example. Their own trace can be shared
with teachers as a basis for feedback on not just the final output, but also the
process by which the list was sorted.

This paper makes the following primary contributions:

\begin{enumerate}

\item The design of a direct manipulation (DM) gesture set for tracing algorithm
behavior on lists, trees and graphs.

\item CodeInk: a CS education tool for tracing algorithms that implements the DM
gesture set and records traces as sharable, interactive program steps.

\item An evaluation of CodeInk's viability in a controlled study, where students
watched CodeInk-produced traces to learn an algorithm, and then used CodeInk to
trace the algorithm on a new example to solidify their understanding.

\end{enumerate}

% \pg{talk broader about future implications for this sort of technique}
While our focus is on enhancing algorithm traces in a CS education context, the
ability to directly manipulate data structures and record those interactions as
program steps may have broader applicability: for the general programmer, it may
enable better visualization and exploration of an algorithm's design. We
conclude this paper with a discussion of future work.

\section{Related Work}

Our work is related to research in code-driven visualization tools, algorithm
visualization, visual programming languages, and direct manipulation user
interfaces.

Code-driven visualization tools enable the user to enter a small program,
single-step through its execution, and see the state of the program at each
step. These tools are powerful for understanding code, but are not ideal for
understanding an algorithm's high-level behavior, since the visualizations occur
at a low level of abstraction: the executed steps are language and
implementation-specific, and the program state is visualized using stack frames
and pointers. CodeInk operates at a higher level of abstraction through its
visual vocabulary of data structures, while its focus on tracing, rather than
visualization, results in a more active learning process.

% Sorva et al.\ provide a comprehensive survey of 44 such tools
% for languages such as Java, C++, and Python~\cite{Sorva2013}.

Algorithm visualizations are custom-built animations of an algorithm's behavior
through which viewers can step backward and forward. The animations are
typically hand-coded using GUI libraries, meaning they are at the ideal level of
abstraction for explaining algorithms (diagrams of data structures, fewer
low-level steps), but are tedious and difficult to author.
CodeInk lowers the barrier to creating algorithm visualizations (no code needs
to be written), while enabling students to learn by both watching and
tracing.

%To enable teachers to find and integrate suitable
%visualizations into their curriculula, Shaffer et al.\ created the AlgoViz
%Portal~\cite{AlgoViz}, a website that catalogues and ranks hundreds of existing
%visualizations.

%Shaffer et al.\ surveyed CS instructors and found that while there was high interest in these visualizations, 
% few used them in practice due to the difficulty of finding and integrating
% suitable visualizations into their curriculum~\cite{Shaffer2011}.

Visual programming languages enable programmers to write programs using
graphical elements rather than by typing text. These languages have gained
widespread adoption in two main areas:
(1) domain-specific languages for specialists, such as LabVIEW for electronic
systems designers and Max/MSP for digital music creators, and (2) educational
programming environments such as Alice~\cite{Alice2008} and
Scratch~\cite{Scratch2008}, which allow novices to create simple programs by
snapping together colorful blocks. CodeInk's DM gesture set is a type of visual
language that focuses on traces of algorithms, rather than general
programs.
% Furthermore, CodeInk's DM language enables descriptions of an algorithm's
% trace, and is not yet a general qprogramming language.

Direct manipulation (DM) user interfaces~\cite{Hutchins1985}, starting with
Shneiderman's initial definiton of the term~\cite{Shneiderman1982}, have long
been recognized to promote more satisfactory reactions among users than
command-line or WIMP interfaces. CodeInk's DM gesture set was designed with
these princples in mind: algorithm steps can be described using physical actions
(grabbing and dragging data structures), lowering the degree of indirection
between onscreen objects and the abstract data structures they represent. For
example, list elements can be rearranged by grabbing elements and moving them
into new positions, rather than by writing code or using menus.

% DM environments for programming by
%demonstration~\cite{Cypher1993} enable users to construct programs by
%demonstrating how the program should behave on specific examples, from which
%% the system refines its understanding of the program. CodeInk's focus is on
% traces of
%algorithms, rather than general programs.


\begin{comment}
Programming by example~\cite{Lieberman2001} enables the user to write programs by
providing example input-output pairs. A related technique, programming by
demonstration~\cite{Cypher1993}, lets the user demonstrate individual steps
using a direct manipulation UI. Tools that embody these techniques often
generalize the user's actions to synthesize programs that operate on new,
unknown inputs~\cite{Yessenov2013, Kandel2011}.
% Common use cases include synthesizing text editing~\cite{Yessenov2013} and
% data cleaning~\cite{Kandel2011} scripts.
CodeInk takes inspiration from these tools and allows the user to explain
algorithms by demonstration, rather than by writing code. CodeInk does not yet
provide generalization capabilities; we plan to add that in future work.
\end{comment}

\section{CodeInk: an algorithm animation tool}


% CodeInk:
%- Objective: enable teachers and students to explain algorithms by directly
% manipulating data structures.
%- Scope: traces and visual problems
%- Usage model through example
%- Use cases

%The primary contribution of this paper is a direct manipulation language
%for explaining algorithms. To realize and evaluate this language, we
%implemented a prototype tool called CodeInk, which serves as our
%%details of our language.

CodeInk (\fig{fig:codeink-intro}) is a Web-based algorithm animation tool that
enables instructors and students to explain algorithms by directly manipulating
data structures. It implements the direct manipulation language we present in the
next section, enabling us to evaluate the language's usability and benefits.
CodeInk's design is motivated by the common convention of teaching and learning
algorithms with worked examples~\cite{Sweller1985} -- step-by-step
demonstrations of how to solve a problem on concrete examples.
\fig{fig:6006-insertion}, for instance, shows how instructors often explain
insertion sort in an introductory algorithms class: they trace the algorithm's
steps on a concrete list rather than, say, analyze abstract pseudocode.
Due to the limitations of the blackboard, the instructor is forced to show
element swaps by redrawing the list in each stage of the sort.

CodeInk eliminates the tedium of drawing such storyboard diagrams, making it possible to
manipulate the data structure directly. Its basic usage model is:

\begin{enumerate}

\item Set up a concrete example of a data structure (list, binary tree,
graph) by dragging constituent objects onto the canvas and
initializing their values.

\item Directly manipulate the data structure using gestures to
demonstrate an algorithm's trace on that data.

\item Every interaction adds a step to the trace, translated into Python
code (\fig{fig:codeink-intro}c). Click on any step to jump to the
corresponding point in the trace, or play back the trace using the
controls in \fig{fig:codeink-intro}b.

\item Share the explanation online by generating a URL, embedding into a
Web page, or exporting as a video.

\end{enumerate}


% Included in tool.tex
\subsection{Direct Manipulation (DM) Gesture Set}

\begin{table}[!b]
% % increase table row spacing, adjust to taste
\renewcommand{\arraystretch}{1.75}
% if using array.sty, it might be a good idea to tweak the value of
% \extrarowheight as needed to properly center the text within the cells
% \setlength{\extrarowheight}{10pt}
\centering
% % Some packages, such as MDW tools, offer better commands for making tables %
% than the plain LaTeX2e tabular which is used here.
\begin{tabular}{|p{3.8cm} |p{3.8cm} |}
\hline
\textbf{Algorithm Behavior} & \textbf{DM Gesture} \\
\hline
%Access a value & Grab value and drag elsewhere \\
%\hline
Use or copy an object's value & {\em drag-away}: Grab an object, and drag it
away quickly.
Drop it on the stage to create a new number with the same value.
\\
\hline
Remove an object from its parent (e.g. element from list, node from tree/graph) & {\em dwell-drag-away}: Grab an object, dwell for one second, then
drag it elsewhere.
\\
\hline
Compare two objects' values & {\em drag-into}: Drag one value into another.
\\
\hline
Assign one object's value to another &
{\em drag-into-dwell}: Drag one value into another, and
dwell until the desired interpretation (\texttt{=, +=, -=}) appears. \\
\hline
Insert a value into a list, or a node into a tree/graph
& {\em drag-insert}: Drag the value into a gap in the list, or the node to the
tip of an edge
\\

\hline
Attach an edge to a graph node
& {\em drag-edge}: Drag the edge's start or end handle to the node.
\\

\hline
Mark a list element or tree/graph node (e.g. as sorted or visited) &
{\em fill}: Select the ``Fill" tool, and click on the element. \\
\hline
\end{tabular}

\caption{CodeInk's Direct Manipulation (DM) Gesture Set}
\label{tbl:gesture_table}

\end{table}

CodeInk's DM gesture set (summarized in \tab{tbl:gesture_table}) enables
users to trace algorithms by manipulating numbers, lists, binary trees and
graphs. From our experience of watching online lecture videos of an introductory
algorithms class, we compiled a list of algorithm behaviors that need to be
expressed when tracing commonly taught sorting and search algorithms (left
column of \tab{tbl:gesture_table}).
For each behavior, we then devised a gesture, in keeping with principles of direct
manipulation~\cite{Shneiderman1982, Lee2012}, that would enable the user to
express each behavior using a physical action. For example, numbers can be
copied by grabbing them and dragging away, and they can be inserted into lists
by dragging them into gaps between other elements.

All data structures in CodeInk's visual vocabulary are composed of one
or more objects (list elements or nodes), each with numeric values. For
that reason, the gesture set is both compact (7 gestures in total) and
expressive: dragging one object into another compares their numeric
values, regardless of whether they are numbers, list elements, nodes or
some mixture thereof. Similarly, popping a list element or detaching a
subtree from its parent is accomplished by grabbing and holding it
(\emph{dwelling}) until it can be moved away freely. We explain the
design decision behind dwelling later in this section. We now illustrate
the gesture set through two examples.

\noindent \textbf{Insertion Sort}: An instructor or student typically
traces insertion sort on an example list by redrawing it in successive
configurations (\fig{fig:6006-insertion}). With CodeInk, the algorithm
can be traced as follows:

\noindent 1) Drag an example list onto the stage (main canvas) and enter
numbers to initialize element values. Drag a finger onto the stage to
point to the starting element. The green arrows below indicate user drag
motion and are not part of the CodeInk UI.

\vspace{-0.25em}
\noindent \includegraphics[width=0.7\columnwidth]{img/examples/insertion-1.png}
\vspace{0.5em}

\noindent 2) To prepare to move the ``2" element, grab and hold the element for
at least one second (\emph{dwell}). A blue circle fills up to give the user
feedback on how long they have dwelled.

\vspace{-0.25em}
\noindent \includegraphics[width=0.7\columnwidth]{img/examples/insertion-2.png}
\vspace{0.4em}

\noindent 3) After one second has elapsed (blue circle filled entirely), the
list expands outward, creating gaps into which the element can be inserted. The
``2" element is now ready to be moved.

\vspace{-0.25em}
\noindent \includegraphics[width=0.7\columnwidth]{img/examples/insertion-3.png}
\vspace{0.5em}

\noindent 4) Drag the ``2" to the left until it hits the ``5" element, which
adds a numeric comparison step (``2 $<$ 5") to the trace.

\vspace{-0.25em}
\noindent \includegraphics[width=0.7\columnwidth]{img/examples/insertion-4.png}
\vspace{0.5em}

\noindent 5) Keep moving the ``2" to the left of the ``5" element.

\vspace{-0.25em}
\noindent \includegraphics[width=0.7\columnwidth]{img/examples/insertion-5.png}
\vspace{0.5em}

\noindent 6) Once the correct position is found, drop the element
to adds a list insertion step to the trace. The list collapses again in its new, rearranged state, with the
``2" preceeding the ``5''.

\noindent \includegraphics[width=0.7\columnwidth]{img/examples/insertion-6.png}

The user traces the rest of insertion sort by moving the finger down the
list and inserting elements into the sorted sublist until the entire
list is sorted. If any mistakes are made, the user can undo steps to
remove them from the trace.


\noindent \textbf{AVL Insertion}: Drawing an AVL (balanced binary search
tree) insertion can be tedious and error-prone, because rotations
require redrawing the entire tree in a new configuration. CodeInk's
gesture set affords users the ability to rearrange nodes in the tree by
detaching, dragging, and reattaching them. As is the case with lists,
dragging one value into another triggers a numeric comparison (Step 2),
and grabbing and dwelling removes a child from its parent (Step 5).


\noindent \begin{tabular}{m{4.6cm} m{3.4cm}}

1) Create two new binary tree nodes (``4" and ``6") by dragging node
objects onto the stage and typing in their numeric values.

& \includegraphics[width=3.4cm]{img/examples/bst-1.png}
\end{tabular}


\noindent \begin{tabular}{m{4.6cm} m{3.4cm}}

2) Drag the ``6'' node into the ``4'' node to compare their values.
CodeInk adds the comparison step (``4 $<$ 6") to the trace.

& \includegraphics[width=3.4cm]{img/examples/bst-2.png}
\end{tabular}

\noindent \begin{tabular}{m{6.2cm} m{1.8cm}}

3) Since ``6" is greater than ``4", keep dragging the ``6'' along the
right pointer of the ``4" node. This gesture temporarily highlights the pointer
blue and adds a \emph{pointer traversal} step to the trace.

& \includegraphics[width=1.8cm]{img/examples/bst-3.png}
\end{tabular}

\noindent \begin{tabular}{m{4.6cm} m{3.4cm}}

4) Keep dragging along the right pointer until reaching its tip. The
``6" node now re-emerges as the right child of ``4." The node is colored blue to
preview the insertion. Release the node to confirm the insertion and
add the step \texttt{binary1.right = binary2} to the trace.

& \includegraphics[width=3.4cm]{img/examples/bst-4.png}
\end{tabular}

\noindent \begin{tabular}{m{4.6cm} m{3.4cm}}

5) Next insert a new ``9" node into the tree, which results in an
unbalanced tree. Now demonstrate a rotation by grabbing the ``6" node,
dwelling for one second, and dragging it away to detach the subtree.
When the subtree is dropped on the stage, a new step is added to the
trace: \texttt{binary1.right = None}

& \includegraphics[width=3.4cm]{img/examples/bst-5.png}
\end{tabular}

\noindent \begin{tabular}{m{4.6cm} m{3.4cm}}

6) The ``6 / 9" subtree is now separated from the ``4." Drag the ``4"
node downward ...

& \includegraphics[width=3.4cm]{img/examples/bst-6.png}
\end{tabular}

\noindent \begin{tabular}{m{4.6cm} m{3.4cm}}

7) ... until it reaches the tip of the ``6" node's left pointer. Then
release to insert it there and balance the tree, thereby adding the step
\texttt{binary2.left = binary1} to the trace.

& \includegraphics[width=3.4cm]{img/examples/bst-7.png}
\end{tabular}

\begin{comment}
\noindent \textbf{Graph algorithms}: Our gesture set also covers graph
traversal, such as search or finding shortest paths (see
\fig{fig:example-dijkstra}). It supports creating and attaching graph
nodes and edges, updating node and edge values, and marking nodes as
visited. Here is how to trace Dijkstra's algorithm using CodeInk:

\noindent \begin{tabular}{m{4.2cm} m{3.8cm}}
1) Create an example graph by dragging and dropping node and edge
objects onto the stage, typing in the numeric node costs and edge weights.
Edges can be connected to nodes by clicking and dragging their start and end
handles.
& \includegraphics[width=3.8cm]{img/examples/dijkstra-1.png}
\end{tabular}

\noindent \begin{tabular}{m{4.2cm} m{3.8cm}}
2) Start with \texttt{node1}. To calculate the
cost of reaching its neighbor \texttt{node2},
first drag \texttt{node1} (source node) and drop it onto the stage, which
creates a new number (\texttt{num1}) equal to the node's cost (\texttt{0}).
& \includegraphics[width=3.8cm]{img/examples/dijkstra-2.png}
\end{tabular}

\noindent \begin{tabular}{m{4.2cm} m{3.8cm}}
3) Now drag the weight of the connecting edge into the current cost
(\texttt{num1}), which triggers a comparison by default. However, a
comparison is not the correct operation; the two values
must be added.
& \includegraphics[width=3.8cm]{img/examples/dijkstra-3.png}
\end{tabular}

\noindent \begin{tabular}{m{4.2cm} m{3.8cm}}
4) Dwelling after the drag-into gesture causes CodeInk to cycle through alternative
interpretations. When an addition assignment
operation (\texttt{+=}) appears, release to end the gesture. The cost
updates to the value of \texttt{10} (\texttt{num1 += edge1.weight}).
& \includegraphics[width=3.8cm]{img/examples/dijkstra-4.png}
\end{tabular}

\noindent \begin{tabular}{m{4.2cm} m{3.8cm}}
5) Now drag \texttt{num1} into \texttt{node2}
to trigger a comparison, checking if the new cost is less than
the existing cost of reaching that node.
& \includegraphics[width=3.8cm]{img/examples/dijkstra-5.png}
\end{tabular}

\noindent \begin{tabular}{m{4.2cm} m{3.8cm}}
6) Since \texttt{10 $<$ $\infty$}, dwell to cycle to the assignment
expression (\texttt{node2.value = num1}). Releasing ends the gesture, and updates the cost
of \texttt{node2} to 10.
& \includegraphics[width=3.8cm]{img/examples/dijkstra-6.png}
\end{tabular}

\noindent \begin{tabular}{m{4.2cm} m{3.8cm}}
7) Repeat on all nodes and mark each one as visited by
selecting ``Fill" and clicking on that node.
& \includegraphics[width=3.8cm]{img/examples/dijkstra-7.png}
\end{tabular}

%Because the drag-into gesture has multiple interpretations
%(comparison, assignment, addition assignment), the user can dwell
%to cycle through all interpretations (see \sec{sec:overloaded-gestures}).
\end{comment}

\subsection{Chaining and traversal patterns}
In CodeInk, the user can chain multiple gestures while dragging an object to
describe a traversal pattern. For example, during insertion sort, the user can
drag an element along the list, performing comparisons without releasing it
until the correct location is found. Similarly, during an AVL insertion, a
candidate node is dragged through the tree to make comparisons and follow
pointers to find the insertion point.

One subtle interaction issue is deciding which behaviors in a chain of
gestures to commit (add) to the trace. For example, the user should be
allowed to click and drag an element into a list, then undo the behavior
by dragging it back out before releasing the mouse button. The insertion
step should be committed only if the new value is released into the
list, but comparisons should be committed as the user is dragging the
element as part of a traversal pattern. We therefore distinguish between
\emph{mutation} and \emph{observation} behaviors as those that affect
the underlying data and those that only visualize decision making in the
algorithm, respectively. In a chain of gestures, observation behaviors
(e.g., comparisons) are continuously committed to the trace, while
mutation behaviors (e.g., insertions) are committed only at the end of a
gesture.

\subsection{Resolving overloaded gestures}
\label{sec:overloaded-gestures}

% When designing a gesture vocabulary, the gesture that seems most
% natural for a task can often be interpreted in multiple ways.
\tab{tbl:gesture_table} shows two overloaded gestures: drag-into and
drag-away. Drag-into is ambiguous because dragging one object into
another could be interpreted as a comparison (\texttt{x$<$y}) or an
assignment (\texttt{x=y}).
% or an augmented assignment such as \texttt{x+=y} or \texttt{x-=y}.
Performing the drag-away gesture on a child object is ambiguous because
it is not obvious whether the parent object should be altered. For
example, when dragging an element away from a list, should a copy of the
element be made, or should the element be popped from the parent list?
% For trees and graphs, should a copy of the dragged node be made, or
% should that node and its descendants be detached from the parent?
%
Our solution defaults to safe observation behaviors and previews other
possible mutation behaviors if the user dwells -- grabs and holds the
object -- for more than one second.
% This means defaulting to a comparison for the drag-into gesture, and
% making a copy of the dragged value for the drag-away gesture.
This design also aligns well with chained traversal patterns, since
comparisons can occur in quick succession while dragging.


\section{Design Rationale: Educational use cases}

Since CodeInk targets instructors and students in an educational
setting, we have designed it to support best practices from educational
psychology: concrete tracing, worked examples, targeted feedback, and
instructional scaffolding.

Novices often need to see many instances of concrete traces on example
data before they can begin to understand how any algorithm works in the
most general case~\cite{Vainio2007}. An instructor can use
CodeInk instead of a blackboard, PowerPoint, or digital drawing
application to record concrete algorithm traces.

Worked examples~\cite{Sweller1985} are step-by-step demonstrations of
how to solve a particular problem. According to cognitive load theory,
novices can learn better by studying how experts solve problems that
would be too difficult for novices to solve on their
own~\cite{Linn1992}. CodeInk allows instructors to easily create worked
examples that explain algorithmic concepts.

Students learn best when they are given targeted feedback that is
as specific as possible to the task at hand~\cite{Balzer1989}. Using
CodeInk, an instructor can set up examples of data structures and then
prompt students to trace the algorithm themselves. Since CodeInk records
all steps and undos, the instructor can see not only whether the final
answer was correct, but also the full history of how the student went
about solving the problem. Thus, they can give targeted feedback on the
student's line of thinking rather than just on the final answer.

Finally, students eventually need to learn to write algorithms in a real
programming language. CodeInk provides a mapping from visual data
structure changes to lines of Python code. This serves as a form of
instructional scaffolding~\cite{Pea2004} to help students acquire
basic programming skills.

\begin{comment}

CodeInk supports three main educational use cases:

\begin{itemize}\itemsep0pt

\item Instructors can easily create algorithm explanations that can be
used in a live class or disseminated online.

\item Students can step through instructor-created explanations to learn
both the algorithms and basic Python constructs, such as list
manipulation operators.

\item Students can solidify their understanding by tracing an algorithm
on new input data. CodeInk records all user interactions, which enables
instructors to give targeted feedback~\cite{Balzer1989} on the student's
problem-solving process.

\end{itemize}

\end{comment}


\section{Initial Evaluation}

% Testing a novel learning interaction, where students watch a
% CodeInk-produced trace of an algorithm, then trace the algorithm for
% themselves on new examples in the same environment
A formative study of our earlier DM gesture set~\cite{Scott2014} showed
that instructors found CodeInk easier to use and more helpful than a
computer drawing application for explaining list sorting algorithms. In
this initial study, we evaluated two hypotheses about CodeInk's
viability for CS students:

\noindent \textbf{H1}: Students learn as effectively from a
CodeInk-produced explanation as they do from a standard lecture video.

\noindent \textbf{H2}: Students are able to correctly trace the learned
algorithm on a new example data structure.

\noindent \textbf{Subjects}: We recruited nine students from the
introductory programming class at our university. Before the study
began, we asked subjects to rate their initial understanding of lists,
insertion sort, and merge sort on a 7-point Likert scale, based on how
well they could explain the concept to another person. The mean age was
20 ($\sigma$=3.12), with 2 males and 7 females. The mean self-reported
understanding for lists was 2.89 ($\sigma$=2.37). 8 out of 9 students
had never seen insertion or merge sort before (mean understanding =
1.22, $\sigma$=0.67).

\noindent \textbf{Tasks and Procedures}: Each subject began the
30-minute study by watching a training video and getting familiar with
CodeInk's UI and DM gesture set. Students then performed the main task
twice, once for insertion sort and once for merge sort. First, they
learned about the algorithm by watching a video ({\em learning phase})
and then explained its trace on a different list back to the
experimenter ({\em explanation phase}).

In the learning phase, the video was either a screencast of a
CodeInk-produced trace or an excerpt from a classroom lecture
video, where an instructor traces the sorting algorithm on an example
list on the blackboard (\fig{fig:6006-insertion}). The CodeInk-produced
trace covered the exact same examples as the lecture videos. After this
phase, the student was asked to rate their new understanding of the
algorithm on a 7-point Likert scale. In the explanation phase, the
student used CodeInk to trace the algorithm on a new list.

% After the main task, subjects filled out an exit questionnaire, rating the
% helpfulness of three key CodeInk features: the direct manipulation language,
% visualizations of comparisons, and having the trace recorded as a list of
% Python steps.

\noindent \textbf{Results}: Our study supported \textbf{H1}. Recall that
8 out of 9 students reported they had never before seen insertion sort
or merge sort. After the learning phase, students reported that they
learned equally well from a CodeInk-produced traces (mean understanding
= 6.00) as they did from a standard lecture video (mean = 5.56). The
difference was not statistically significant (p=0.196 using a
Mann-Whitney U test). \textbf{H2} was also supported, because all
students who watched the CodeInk-produced trace could then use CodeInk
to trace both insertion sort and merge sort correctly on the new list.

%That is, both the process and final output demonstrated in their traces
%were correct.

%In other words, we found no evidence that CodeInk is any more or less effective
%than a standard online lecture video for teaching these particular algorithms.

This initial study suggests the potential for a new kind of learning
workflow, where CodeInk explanations can be embedded in online course
content. Students can learn about algorithms and practice tracing their
behavior in the same environment. Their traces could then be shared and
analyzed by teaching staff, manually or automatically, as a basis for
targeted feedback~\cite{Balzer1989} on the student's understanding.


\section{Design Space and Future Work}
\label{sec:design-and-future-work}

CodeInk's DM language represents one point in the design space of languages for
programming by direct manipulation. We describe two dimensions of this design
space -- coverage and generalization -- to frame CodeInk's contributions,
limitations, and directions for future work.

\subsection{Coverage of Data Structures and Behaviors}

Programs written for domains ranging from classrooms to professional
software engineering require varying coverage of data structures and
algorithms. Even within CS education, introductory
algorithms classes and popular algorithms textbooks such as
CLRS~\cite{Cormen2001} cover a wide range of problem types. Some algorithms,
such as numerical analysis algorithms, are inherently not visual and may
not benefit from visual and direct manipulation programming languages.

CodeInk's DM language supports sorting and search across lists, binary
trees, and graphs. These algorithms cover introductory-level material
and comprise roughly one third of the algorithms in
CLRS~\cite{Cormen2001}: insertion sort, heapsort, mergesort, quicksort,
operations on binary search trees and red-black trees, breadth-first and
depth-first graph traversal, and the Bellman-Ford and Dijkstra's
shortest-path algorithms.

To provide completeness for the educational domain, we plan to extend
CodeInk's language to cover additional concepts in CLRS and in advanced
CS courses. Supporting strings, linked lists, hash tables, and tabular
data structures (e.g., 2D arrays) will enable explanations of more
sophisticated data structures, dynamic programming, and matrix
algorithms.

Second, this work focuses on supporting traces on lists, binary trees, and
graphs, because their canonical algorithms -- sorting, rotation, search and
shortest paths -- are widely taught in introductory CS and algorithms classes.
These data structures and their associated algorithms are also inherently
visual: The natural way to teach and think through them is by drawing diagrams.
Thus, we envision direct manipulation for tracing to be most useful for these
types of problems.

\subsection{Generalization}
There is a distinction between a DM language for explaining an
algorithm's trace on an example, and a DM language for describing general programs.
CodeInk's current focus is on the former, which is well-suited for teaching
algorithms via worked examples~\cite{Sweller1985}. However, we plan to extend
its language from 
demonstrations of concrete data changes to expressing flow
control and functions.
We are now prototyping a direct manipulation vocabulary for describing iteration
sequences and branching conditions on example data structures. We plan to
bring CodeInk beyond descriptions of algorithm traces, in order to make it an
environment for programming by direct manipulation.

%We now show CodeInk being used to describe three different algorithms,
%in order to show demonstrate the coverage of its direct manipulation
%vocabulary. Specifically, we demonstrate explanations of merge sort (a
%list sorting algorithm), a binary search tree insertion and rotation
%(or an AVL insertion), and Dijkstra's algorithm for finding the
%shortest path in a graph.

%\subsubsection{Merge Sort}
%\includegraphics[width=\pagewidth]{mergesort}
%\subsubsection{AVL Insertion}
%\includegraphics[width=\pagewidth]{BST}
%\subsubsection{Dijkstra's Algorithm}


\section{Conclusion}

This paper presents the design of \emph{CodeInk}, a system for
tracing algorithms by direct manipulation (DM) of data structures. Its
DM gesture set enables users to transform lists, trees, and graphs,
rather than having to tediously draw storyboards of an algorithm's
behavior. The user-described trace is recorded
as interactive program steps, in Python or explanatory English, which
enables navigation through the trace, and which can be shared and used as a
basis for targeted feedback. In an initial study that evaluated
CodeInk's viability as a tool in CS education, we found that students
were able to learn as effectively from a CodeInk-produced explanation as
from standard lecture videos, and that they could use CodeInk to trace
the learned algorithm on new example data structures.

%We plan to extend the coverage of CodeInk's vocabulary to more data
%structures and algorithms. We are also prototyping gestures for flow
%control as a step toward an environment for describing general programs,
%not just concrete traces, by direct manipulation.


% use section* for acknowledgement

%\section*{Acknowledgments} The authors would like to thank Rob Miller,
%Tom Lieber, Juho Kim, Carrie Cai, Elena Glassman, the UID group at MIT
%CSAIL and all our study participants for their feedback. This work was
%funded in part by Quanta Computer and NSF. \todo{put the exact grant
%number from the COUHES form}

% trigger a \newpage just before the given reference
% number - used to balance the columns on the last page
% adjust value as needed - may need to be readjusted if
% the document is modified later
%\IEEEtriggeratref{8}
% The "triggered" command can be changed if desired:
%\IEEEtriggercmd{\enlargethispage{-5in}}

% references section

% can use a bibliography generated by BibTeX as a .bbl file
% BibTeX documentation can be easily obtained at:
% http://www.ctan.org/tex-archive/biblio/bibtex/contrib/doc/
% The IEEEtran BibTeX style support page is at:
% http://www.michaelshell.org/tex/ieeetran/bibtex/
\bibliographystyle{IEEEtran}
% argument is your BibTeX string definitions and bibliography database(s)
\bibliography{codeinkVLHCC}
%
% <OR> manually copy in the resultant .bbl file
% set second argument of \begin to the number of references
% (used to reserve space for the reference number labels box)
%\begin{thebibliography}{1}
%
%\bibitem{IEEEhowto:kopka}
%H.~Kopka and P.~W. Daly, \emph{A Guide to \LaTeX}, 3rd~ed.\hskip 1em plus
%  0.5em minus 0.4em\relax Harlow, England: Addison-Wesley, 1999.
%
%\end{thebibliography}


% that's all folks
\end{document}

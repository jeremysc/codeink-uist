\section{Introduction}

%Tracing algorithms is ubiquitous, fundamental
Tracing algorithms on concrete examples is fundamental to how teachers and students in
computer science explain and learn, respectively. On the blackboard, teachers explain an
algorithm for the first time by drawing an example data structure and carrying
out the algorithm's operation on the example. Similarly, before
attempting problem sets and coding assignments,
students often review algorithms by reading pseudocode and
tracing through multiple examples on paper~\cite{Vainio2007}.
% mentally tracing the algorithm's behavior

% Tedious, limiting no manipulation affordances no persistent, structured
% recording
This process of tracing algorithms on blackboards and paper can be both tedious
and limiting. First, blackboards and paper do not afford manipulation of the
drawing. To demonstrate changes, the user must erase, redraw, and storyboard the
behavior as a series of static snapshots.
Second, the drawing process is not recorded in a persistent, structured format,
so it can be hard for students and teachers to share ideas and discuss problems
unless they are co-located.
This limitation is particularly problematic in a MOOC context, where the
majority of students do not have direct access to teaching staff.

\begin{figure}

\begin{center}
%\includegraphics[width=0.55\columnwidth]{img/frontpage-6006.png}
\includegraphics[width=\columnwidth]{img/frontpage-mergesort.png}
\end{center}

\caption{CodeInk is a Web-based tool that implements a direct
manipulation gesture set for tracing algorithms on data structures for the
purpose of CS education. CodeInk allows the user to (a) compose and
manipulate data structures on a graphical canvas, (b) step forward and
backward through the recorded trace, and (c) see each step translated
into a line of Python code or an English explanation.}

%The CodeInk user interface is (a) a canvas for composing and
%manipulating data structures and (c) an interactive list of Python steps
%recorded by the tool, based on the user's interactions. The user can
%step forward and backward through the steps, or replay the entire trace
%using the playback controls (b).}

\label{fig:codeink-intro}
\end{figure}

% Code-driven visualization is for visualization, not explanation or active
% learning Also at the wrong level of abstraction
Code-driven program visualizations~\cite{Guo2013, Sorva2013} are an alternative
to manually drawing traces on blackboards or paper. Using such a tool, the user
can enter code that implements an algorithm and then watch a step-by-step
animation of its execution.
However, the implementation of an algorithm is usually at a lower level of
abstraction than the language-agnostic level at which the algorithm should be
explained and learned. Also, watching a visualization is passive learning,
whereas tracing requires the student to play the role of the computer and carry
out the algorithm's behavior.
This form of active learning has been shown to result in better learning
outcomes~\cite{Sorva2012Diss}.

We hypothesize that the ideal user interface for teaching and learning
algorithms should support exploration by tracing, and enhance the experience by
(a) affording users the ability to directly manipulate data structures, and
(b) recording the trace in a structured, persistent format that affords
dissemination, interaction and discussion.

To explore this hypotheiss, we built \emph{CodeInk}, a Web-based tool for
tracing algorithms by direct manipulation. It includes a novel direct
manipulation (DM) gesture set that enables users to transform lists, trees and
graphs rather than tediously draw before-and-after snapshots. In CodeInk, when
the user manipulates data structures, a trace is recorded as interactive program
steps, where each gesture is translated to a line of Python code or explanatory
English. The steps provide three main benefits: feedback on user interactions,
navigation through the trace, and a format that can be easily disseminated and
analyzed as a basis for discussion.

\fig{fig:codeink-intro} shows how an instructor can explain the merge sort
algorithm in CodeInk by dragging an example list onto the canvas, selecting
sublists with a rectangular selection, dragging them away to create copies, then
merging elements by dragging them into a new sorted list
(\fig{fig:codeink-intro}a).
Every interaction is interpreted as a step in Python (\fig{fig:codeink-intro}c).
The trace can then be shared with students, who can navigate through it by
single-stepping or clicking on steps (\fig{fig:codeink-intro}b), and then trace
the algorithm for themselves on a new example. Their own trace can be shared
with teachers as a basis for feedback on not just the final output, but also the
process by which the list was sorted.

This paper makes the following primary contributions:

\begin{enumerate}

\item The design of a direct manipulation (DM) gesture set for tracing algorithm
behavior on lists, trees and graphs.

\item CodeInk: a CS education tool for tracing algorithms that implements the DM
gesture set and records traces as sharable, interactive program steps.

\item An evaluation of CodeInk's viability in a controlled study, where students
watched CodeInk-produced traces to learn an algorithm, and then used CodeInk to
trace the algorithm on a new example to solidify their understanding.

\end{enumerate}

% \pg{talk broader about future implications for this sort of technique}
While our focus is on enhancing algorithm traces in a CS education context, the
ability to directly manipulate data structures and record those interactions as
program steps may have broader applicability: for the general programmer, it may
enable better visualization and exploration of an algorithm's design. We
conclude this paper with a discussion of future work.

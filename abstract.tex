\begin{abstract}

% 1-2 sentences: motivation
Tracing algorithms on concrete examples is fundamental to how instructors explain
and how students learn an algorithm's behavior. On both blackboards and paper,
people typically draw an example data structure and a storyboard showing how it�s transformed by
the algorithm. But hand drawing is tedious and limiting
because data structures must be erased and redrawn, and the
step-by-step trace of the algorithm cannot be easily shared or discussed.
% 1 sentence: hypothesis / general idea
We present CodeInk, a Web-based system for tracing algorithms that enables the
user to directly manipulate data structures and that records the actions as a
set of interactive, sharable steps.
% 1 sentence: how things work / how the contributions were achieved
The system's direct manipulation gesture set enables the user to demonstrate the algorithm as a
process, instead of drawing static snapshots of its effects.
All interactions are recorded as steps in Python or English, which
enable navigation through the trace. Steps can also be shared or analyzed as a basis for
feedback from instructors to students.
% 1-2 sentences: evaluation results 1 sentence: this paper describes,
% contributions
In a user study, we found that students learned equally well from
CodeInk-produced traces as they did from lecture videos, and that they
were then able to trace the algorithm for themselves in CodeInk.

\end{abstract}

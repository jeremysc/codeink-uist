\begin{abstract}

% 1-2 sentences: motivation
Tracing algorithms on concrete examples is fundamental to how teachers explain
and how students learn an algorithm's behavior. On both blackboards and paper,
they draw an example data structure and a storyboard of how it is transformed by
the algorithm. Unfortunately, drawing in this way can be tedious and limiting
because the data structures can be only erased and redrawn, and the trace cannot
be easily shared or discussed.
% 1 sentence: hypothesis / general idea
We present CodeInk, a novel CS education tool that enhances the experience of
tracing algorithms by enabling users to directly manipulate (DM) data structures
and record the trace as a set of interactive program steps.
% 1 sentence: how things work / how the contributions were achieved
The system's DM gesture set enables the user to demonstrate the algorithm as a
process, instead of drawing static snapshots of its effects.
All interactions are recorded as steps in Python or explanatory English, which
enable navigation through the trace and can be shared or analyzed as a basis for
feedback from teachers to students.
% 1-2 sentences: evaluation results 1 sentence: this paper describes,
% contributions
In a controlled study, we found that students learned equally well from
CodeInk-produced explanations as they did from lecture videos, and that they
were then able to trace the algorithm for themselves in CodeInk. 

\end{abstract}

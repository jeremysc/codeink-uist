\begin{abstract}

% 1-2 sentences: motivation
Tracing algorithms on examples is fundamental to how teachers explain and how
students learn an algorithm's behavior; on blackboards and paper, they draw an
example data structure and a storyboard of how it is transformed by the
algorithm. Unfortunately, drawing in this way can be tedious and limiting
because the data structures can only be erased and redrawn, and the trace cannot
be easily shared or discussed.
% 1 sentence: hypothesis / general idea
We present CodeInk, a novel CS education tool that enhances the experience of
tracing algorithms by enabling users to directly manipulate data structures and
record the trace as a set of program steps.
% 1 sentence: how things work / how the contributions were achieved
The system's DM gesture set enables the user to demonstrate the algorithm as a
process, instead of drawing snapshots of its effects.
In response, the system translates all interactions into a set of Python or
English-explanation steps, which enable navigation through the trace and can be
shared or analyzed as a basis for feedback from teachers to students.
% 1-2 sentences: evaluation results 1 sentence: this paper describes,
% contributions
To evaluate CodeInk's viability as an educational tool, we performed a
controlled experiment in which students from an undergraduate programming class
were asked to learn list sorting algorithms from CodeInk-produced traces, then
trace the algorithms for themselves on new example lists.


%Explaining algorithms is a fundamental part of teaching computer
%science, but instructors find it tedious to draw diagrams on
%the board or in software such as PowerPoint. The shapes in such diagrams
%have no programming-specific affordances or constraints, so they
%provide no scaffolding for students to translate between diagrams and
%code.
%
%In response, we developed a
%direct manipulation (DM) language for explaining algorithms and built a tool
%called CodeInk that implements this language. Our DM language maps gestures
% onto primitive program behaviors that occur in commonly taught algorithms.
%CodeInk enables teachers and students to trace an
%algorithm step-by-step by directly manipulating an example data structure.
%Every interaction step is recorded as a line of Python code.
%This paper describes CodeInk and the design of its DM language.
%It also presents a comparative
%study of CodeInk on 9 students and 4 teaching assistants from introductory
% computer science classes. Subjects found CodeInk easier to use and more helpful than a
%standard drawing application for explaining list sorting algorithms.

\end{abstract}

\begin{abstract}

Explaining algorithms is a fundamental part of teaching computer
science, but instructors find it tedious to draw diagrams on
the board or in software such as PowerPoint. The shapes in such diagrams
have no programming-specific affordances or constraints, so they
provide no scaffolding for students to translate between diagrams and
code.
%
In response, we developed a
direct manipulation (DM) language for explaining algorithms and built a tool
called CodeInk that implements this language. Our DM language maps gestures onto
primitive program behaviors that occur in commonly taught algorithms.
CodeInk enables teachers and students to trace an
algorithm step-by-step by directly manipulating an example data structure.
Every interaction step is recorded as a line of Python code.
This paper describes CodeInk and the design of its DM language.
It also presents a comparative
study of CodeInk on 9 students and 4 teaching assistants from introductory computer
science classes. Subjects found CodeInk easier to use and more helpful than a
standard drawing application for explaining list sorting algorithms.

%1-2 sentences: motivation
%1 sentence: hypothesis / general idea
%1 sentence: how things work / how the contributions were achieved
%1-2 sentences: evaluation results
%1 sentence: this paper describes, contributions
\end{abstract}

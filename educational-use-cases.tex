\section{Educational use cases}

We envision CodeInk being used by instructors and students in educational
settings, particularly as part of online curricula. Thus, we have designed
CodeInk to support certain best practices from the educational psychology
literature, including concrete traces, worked examples, targeted feedback and
instructional scaffolding.

Novices often need to see many examples of concrete traces before being able to
understand how an algorithm works in the most general case~\cite{Detienne90}. An
instructor can use CodeInk instead of a blackboard, PowerPoint, or a digital
drawing application to record concrete traces of an algorithm running on example
data structures.


Worked examples~\cite{Sweller1985} are step-by-step demonstrations of
how to solve a particular problem. According to cognitive load theory,
novices can learn better by studying how experts solve problems that
would be too difficult for novices to solve on their
own~\cite{Linn1992}. CodeInk makes it easier to create worked examples
to explain algorithms at a conceptual level.

Students learn best when they are given targeted feedback that is
as specific as possible to the task at hand~\cite{Balzer1989}. Using
CodeInk, an instructor can set up examples of data structures and then
prompt students to trace the algorithm themselves. Since CodeInk records
all steps and undos, the instructor can see not only whether the final
answer was correct, but also the full history of how the student went
about solving the problem. Thus, they can give targeted feedback on the
student's line of thinking rather than just on the final answer.

Finally, students eventually need to learn to write algorithms in a programming
language. CodeInk provides a mapping from visual data structure changes to lines
of Python code. This serves as a form of instructional
scaffolding~\cite{Pea2004} to help students acquire programming skills.

%Worked examples are most effective when students attempt to explain the
%solution to themselves (i.e.,
%\emph{self-explanation}~\cite{Sandoval95}), but researchers have found
%that beginners are often not inclined to perform
%self-explanations~\cite{Chi1989}.


CodeInk has three potential benefits for CS education:

\begin{itemize}\itemsep2pt

\item Instructors can easily create algorithm explanations that can be
used in class or disseminated online.

\item Students can step through instructor-created explanations to learn
both the algorithms and basic Python constructs, such as list
manipulation operators.

\item Students can solidify their understanding by tracing an algorithm on new
input data. CodeInk records all user interactions, which enables instructors to
give targeted feedback~\cite{Balzer1989} on the student's problem-solving
process.

\end{itemize}
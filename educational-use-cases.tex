\section{Design Rationale: Educational use cases}


\begin{figure}
\begin{center}
\includegraphics[width=0.7\columnwidth]{img/6006/insertion.png}
\end{center}

\caption{An algorithms instructor traces the operation of insertion sort 
on a concrete example of a list by drawing storyboard snapshots.}

\label{fig:6006-insertion}
\end{figure}


Since CodeInk targets instructors and students, its design is inspired
by two main sources: (1) watching popular online lecture videos of
instructors tracing algorithms on blackboards, and (2) best practices
from educational psychology.

\noindent \textbf{Concrete traces}:
We observed that instructors often explain algorithms by drawing concrete examples of
data structures and then tracing the algorithm's behavior on those
examples (\fig{fig:6006-insertion}). With a \emph{concrete trace}, the instructor does not
explain the algorithm in the general case or graphically draw flow control.
Instead, they unfold the loops and branches of an algorithm's pseudocode
into a linear sequence of steps.
% This approach is in line with findings that students need to see multiple
% examples of an algorithm's behavior to build their
% understanding~\cite{Vainio2007}.
CodeInk targets the task of tracing algorithms through concrete examples, rather
than explaining them in the abstract.

%Also, according to cognitive load theory, novices can learn better by watching
%experts solve problems that would be too difficult for novices to solve on
%% their own~\cite{Linn1992, Sweller1985}. The instructor's worked examples give
%% students
%an opportunity to get the gist of an algorithm, before they attempt to trace it
%on their own. 

\noindent \textbf{Level of abstraction}: The instructor's explanation
occurs at a high level of graphical abstraction. They rarely write
language-specific code on the board (e.g., in Python) and will instead
opt for a combination of pseudocode and data structure diagrams. Their
drawings abstract away language-specific details that are not essential
to understanding the essence of the algorithm. We designed CodeInk to
facilitate tracing by direct manipulation of data structure diagrams.
% This should not be confused with the previous point: the traced steps
% are concrete, and do not include control abstraction (e.g. loops).


\begin{figure}
\begin{center}
\includegraphics[width=0.8\columnwidth]{img/visual-vocabulary.png}
\end{center}

\caption{CodeInk's visual vocabulary and data palette, located at the
top of the interface. The green arrows are for illustration only, and
show how the user can drag data structures (numbers, lists, binary tree
nodes, graph nodes, edges) and visual fingers onto the stage.}

\label{fig:visual_vocab}
\end{figure}

\noindent \textbf{Visual vocabulary}: When drawing data structures,
instructors use a familiar, consistent visual vocabulary. Lists are
drawn as a row of boxes (\fig{fig:6006-insertion}), and trees and graphs
are drawn as circles connected by arrows. Values are drawn as a number
within or adjacent to each object. All examples in the
lecture videos we watched were numerical. CodeInk's visual vocabulary
(\fig{fig:visual_vocab}) mirrors these conventions.

\begin{comment}
Instructors add to this vocabulary using arrows and colors. Arrows are
interchangably used to show transitions between storyboard frames, or to point
to the current focus of the algorithm (\fig{fig:6006-insertion}). The former is
an artifact of the need to draw snapshots when using blackboards, while the
latter is an important visual cue for following the algorithm's progress. Colors
are also used to visualize state; for example, to mark graph nodes as visited,
they are filled with a different color. CodeInk's visual vocabulary includes
\emph{fingers}, which can be attached to any on-screen object, and a \emph{fill}
gesture (explained later) that can be used to color a data structure.
\end{comment}

\noindent \textbf{Student-produced traces}: Lister et
al.~\cite{Lister2004} found that a student's ability to trace code is a
prerequisite for being adept at problem solving and writing code. They
stated that CS courses should ``first teach systematic tracing as a base
skill, then allow students to build [\ldots]\ upon that''.
% found that students in introductory programming courses could not
% consistently demonstrate an understanding of code by manually tracing
% it. The researchers noted that ``even when our principal aim is to
% teach students to write code, we require students to learn by reading
% code''
In addition to being an authoring tool for instructors, CodeInk enables
students to practice tracing algorithmic pseudocode on examples. The
student has the freedom to explore an algorithm's behavior, perhaps
making mistakes along the way. CodeInk records the student's full
traces, which allows instructors to provide precise, targeted
feedback~\cite{Balzer1989} on the final answer and the process by which
the student reached their answer.

% Show your work

%Finally, students eventually need to learn to write algorithms in a real
%programming language. CodeInk provides a mapping from visual data
%structure changes to lines of Python code. This serves as a form of
%instructional scaffolding~\cite{Pea2004} to help students acquire
%basic programming skills.

\begin{comment}

CodeInk supports three main educational use cases:

\begin{itemize}\itemsep0pt

\item Instructors can easily create algorithm explanations that can be
used in a live class or disseminated online.

\item Students can step through instructor-created explanations to learn
both the algorithms and basic Python constructs, such as list
manipulation operators.

\item Students can solidify their understanding by tracing an algorithm
on new input data. CodeInk records all user interactions, which enables
instructors to give targeted feedback~\cite{Balzer1989} on the student's
problem-solving process.

\end{itemize}

\end{comment}


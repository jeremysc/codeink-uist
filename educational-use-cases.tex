\section{Design Rationale: Educational use cases}

Since CodeInk targets instructors and students in an educational
setting, we have designed it to support best practices from educational
psychology: concrete tracing, worked examples, targeted feedback, and
instructional scaffolding.

Novices often need to see many instances of concrete traces on example
data before they can begin to understand how any algorithm works in the
most general case~\cite{Vainio2007}. An instructor can use
CodeInk instead of a blackboard, PowerPoint, or digital drawing
application to record concrete algorithm traces.

Worked examples~\cite{Sweller1985} are step-by-step demonstrations of
how to solve a particular problem. According to cognitive load theory,
novices can learn better by studying how experts solve problems that
would be too difficult for novices to solve on their
own~\cite{Linn1992}. CodeInk allows instructors to easily create worked
examples that explain algorithmic concepts.

Students learn best when they are given targeted feedback that is
as specific as possible to the task at hand~\cite{Balzer1989}. Using
CodeInk, an instructor can set up examples of data structures and then
prompt students to trace the algorithm themselves. Since CodeInk records
all steps and undos, the instructor can see not only whether the final
answer was correct, but also the full history of how the student went
about solving the problem. Thus, they can give targeted feedback on the
student's line of thinking rather than just on the final answer.

Finally, students eventually need to learn to write algorithms in a real
programming language. CodeInk provides a mapping from visual data
structure changes to lines of Python code. This serves as a form of
instructional scaffolding~\cite{Pea2004} to help students acquire
basic programming skills.

\begin{comment}

CodeInk supports three main educational use cases:

\begin{itemize}\itemsep0pt

\item Instructors can easily create algorithm explanations that can be
used in a live class or disseminated online.

\item Students can step through instructor-created explanations to learn
both the algorithms and basic Python constructs, such as list
manipulation operators.

\item Students can solidify their understanding by tracing an algorithm
on new input data. CodeInk records all user interactions, which enables
instructors to give targeted feedback~\cite{Balzer1989} on the student's
problem-solving process.

\end{itemize}

\end{comment}


\section{Limitations and Future Work}
\label{sec:design-and-future-work}

CodeInk's DM language supports traces of sorting and search algorithms
across lists, binary trees, and graphs. These algorithms comprise
roughly one third of the algorithms in CLRS~\cite{Cormen2001}:
comparison sorts, operations on binary search and red-black trees,
breadth-first and depth-first graph traversal and some shortest path
algorithms. To provide completeness for the educational domain, we plan
to extend CodeInk's vocabulary to support additional data structures
(strings, linked lists, hash tables, and 2D arrays), as well as compound
data structures (e.g., lists of nodes). Doing so would enable tracing a
broader class of algorithms (e.g., dynamic programming, matrix
operations) and algorithms on compound data structures (e.g., a queue of
nodes in a breadth-first traversal).

Finally, there is a distinction between tracing algorithms and
describing general programs. CodeInk's current focus is on the former,
which is well-suited for teaching algorithms via concrete traces.
However, it is essential for students to build an understanding of
control abstraction: iteration sequences, branching conditions, and
recursion. We plan to extend CodeInk's vocabulary to include gestures
for specifying flow control. We are now prototyping gestures for
describing iteration by tapping objects and controlling the loop's
execution by dragging a slider between those objects. Our vision is to
bring CodeInk beyond algorithm traces to bridge the gap from concrete
examples to abstract program behavior.


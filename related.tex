\section{Related Work}

Code-driven visualization tools~\cite{Guo2013, Sorva2013} generate step-by-step animations of code snippet
execution. While powerful for understanding snippets, they
are not as good for understanding an algorithm's high-level behavior, because they work at the wrong level of abstraction � i.e., the {\em implementation} of the algorithm in a particular language. Algorithm
visualizations~\cite{AlgoViz}, by contrast, operate at the
proper level of abstraction, but must be hand-coded for each 
algorithm. 

CodeInk offers an alternative, low-effort method of authoring
visualizations and enables students to learn by playing the role of the computer, 
actively tracing algorithm behavior. 
CodeInk is thus an example of active learning, which has been shown
to result in better pedagogical outcomes~\cite{Sorva2012Diss}.

% Sorva et al.\ provide a comprehensive survey of 44 such tools
% for languages such as Java, C++, and Python~\cite{Sorva2013}.

%To enable teachers to find and integrate suitable
%visualizations into their curriculula, Shaffer et al.\ created the AlgoViz
%Portal~\cite{AlgoViz}, a website that catalogues and ranks hundreds of existing
%visualizations.

%Shaffer et al.\ surveyed CS instructors and found that while there was high interest in these visualizations, 
% few used them in practice due to the difficulty of finding and integrating
% suitable visualizations into their curriculum~\cite{Shaffer2011}.

Visual programming languages share a similar spirit, in that they enable writing programs by directly
manipulating graphical elements rather than by typing text. These languages have
gained widespread adoption in 
(1) domain-specific languages for technical specialists (e.g. LabVIEW, Max/MSP),
% such as LabVIEW for electronic systems designers and Max/MSP for digital music
% creators
and (2) educational programming environments such as Alice~\cite{Alice2008} and
Scratch~\cite{Scratch2008}, which allow novices to create simple programs by
snapping together blocks. CodeInk introduces a new visual
language for education that enables users to trace algorithms by direct
manipulation of data structures.

\begin{comment}
Direct manipulation (DM) user interfaces~\cite{Hutchins1985}, starting with
Shneiderman's initial definiton of the term~\cite{Shneiderman1982}, have long
been recognized to promote more satisfactory reactions among users than
command-line or WIMP interfaces. CodeInk's DM gesture set was designed with
these princples in mind: algorithm steps can be described using physical actions
(grabbing and dragging data structures), lowering the degree of indirection
between onscreen objects and the abstract data structures they represent. For
example, list elements can be rearranged by grabbing elements and moving them
into new positions, rather than by writing code or using menus.
\end{comment}

% DM environments for programming by
%demonstration~\cite{Cypher1993} enable users to construct programs by
%demonstrating how the program should behave on specific examples, from which
%% the system refines its understanding of the program. CodeInk's focus is on
% traces of
%algorithms, rather than general programs.


\begin{comment}
Programming by example~\cite{Lieberman2001} enables the user to write programs by
providing example input-output pairs. A related technique, programming by
demonstration~\cite{Cypher1993}, lets the user demonstrate individual steps
using a direct manipulation UI. Tools that embody these techniques often
generalize the user's actions to synthesize programs that operate on new,
unknown inputs~\cite{Yessenov2013, Kandel2011}.
% Common use cases include synthesizing text editing~\cite{Yessenov2013} and
% data cleaning~\cite{Kandel2011} scripts.
CodeInk takes inspiration from these tools and allows the user to explain
algorithms by demonstration, rather than by writing code. CodeInk does not yet
provide generalization capabilities; we plan to add that in future work.
\end{comment}

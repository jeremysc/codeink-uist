\section{Related Work}

Our work is related to research in algorithm visualization tools,
program visualization tools, visual programming languages, and programming by
example.

Algorithm visualization tools enable the user to enter their own input data,
single-step through the execution of an algorithm, and see its visual state at
each step. But the barrier to creating such visualizations is high, since each
must be hand-coded using GUI libraries. Shaffer et al.\ surveyed CS instructors
and found that while there was high interest in these visualizations, few
actually used them in practice due to the difficulty of finding and integrating
suitable visualizations into their curriculum~\cite{Shaffer2011}. In response,
Shaffer et al.\ created the AlgoViz Portal~\cite{AlgoViz}, a website that
catalogues and ranks hundreds of existing visualizations. CodeInk takes a
complementary approach by lowering the barrier to creating algorithm
visualizations. Instructors can easily create their own rather than needing to
find and integrate ones written by others.

% hundhausen -- active engagement with vis is important

Program visualization tools enable the user to enter a small program,
single-step through its execution, and see its visual state at each step. Sorva
et al.\ provide a comprehensive survey of 44 such tools for
languages such as Java, C++, and Python~\cite{Sorva2013}. In contrast to algorithm visualization, these tools
visualize the low-level semantics of a specific programming language using
constructs such as stack frames and pointers. CodeInk operates at a higher level
of abstraction by providing a visual language for data structures and a DM
language for describing algorithms.

%\todo{what about CodeInk backend using OPT to generate code?}
%\cite{Guo2013}

Visual programming languages enable programmers to write programs via
direct manipulation of graphical elements rather than by typing text. These
languages have gained widespread adoption in two main areas:
(1) domain-specific languages for specialists, such as LabVIEW for electronic
systems designers and Max/MSP for digital music creators, and (2) educational
programming environments such as Alice~\cite{Alice2008} and
Scratch~\cite{Scratch2008}, which allow novices to create simple programs by
snapping together colorful blocks. CodeInk is also an educational tool, but its
focus is on algorithms rather than basic programming concepts.
% Furthermore, CodeInk's DM language enables descriptions of an algorithm's
% trace, and is not yet a general programming language.

Programming by example~\cite{Lieberman2001} enables the user to write programs by
providing example input-output pairs. A related technique, programming by
demonstration~\cite{Cypher1993}, lets the user demonstrate individual steps
using a direct manipulation UI. Tools that embody these techniques often
generalize the user's actions to synthesize programs that operate on new,
unknown inputs~\cite{Yessenov2013, Kandel2011}.
% Common use cases include synthesizing text editing~\cite{Yessenov2013} and
% data cleaning~\cite{Kandel2011} scripts.
CodeInk takes inspiration from these tools and allows the user to explain
algorithms by demonstration, rather than by writing code. CodeInk does not yet
provide generalization capabilities; we plan to add that in future work.

%(see \sec{sec:design-and-future-work}).


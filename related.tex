\section{Related Work}

Our work is related to research in algorithm visualizations, program
visualization tools, direct manipulation interfaces, and visual
programming languages.

Algorithm visualizations enable the user to enter their own input data,
single-step through the execution of an algorithm, and see its visual
state at each step. The barrier to creating such visualizations is high,
since each one must be hand-coded using a programming language and its
GUI libraries. Shaffer et al.\ surveyed CS instructors and found that
while there was high interest in these visualizations, few actually used
them in practice due to the difficulty of finding and integrating
suitable visualizations into their curriculum~\cite{Shaffer2011}. In
response, Shaffer et al.\ created the AlgoViz Portal~\cite{AlgoViz}, a
website that catalogues and ranks hundreds of existing visualizations.
CodeInk takes a complementary approach by lowering the barrier to
creating algorithm visualizations. No code needs to be written to create
visualizations with CodeInk.

Code-driven program visualization tools enable the user to enter a small
program, single-step through its execution, and see its visual state at
each step. Sorva et al.\ provide a comprehensive survey of 44 such tools
for languages such as Java, C++, and Python~\cite{Sorva2013}. In
contrast to algorithm visualization, these tools visualize the low-level
semantics of a specific programming language using constructs such as
stack frames and pointers. CodeInk operates at a higher level of
abstraction by providing a direct manipulation gesture set for tracing
algorithms on data structures.

Direct manipulation (DM) user interfaces~\cite{Hutchins1985} have a long
history in HCI research. Starting with seminal work such as Sutherland's
Sketchpad~\cite{Sutherland1963}, researchers have been making DM
interfaces for programming for the past fifty years. In particular,
programming by demonstration~\cite{Cypher1993} lets the user construct
domain-specific programs by demonstrating steps using a direct
manipulation UI. CodeInk takes inspiration from these systems and allows
educators to trace algorithms by demonstration rather than by writing
code.

Visual programming languages enable programmers to write programs via
direct manipulation of graphical elements rather than by typing text. These
languages have gained widespread adoption in two main areas:
(1) domain-specific languages for specialists, such as LabVIEW for electronic
systems designers and Max/MSP for digital music creators, and (2) educational
programming environments such as Alice~\cite{Alice2008} and
Scratch~\cite{Scratch2008}, which allow novices to create simple programs by
snapping together colorful blocks. CodeInk is also an educational tool, but its
focus is on algorithms rather than on introductory programming concepts.
% Furthermore, CodeInk's DM language enables descriptions of an algorithm's
% trace, and is not yet a general programming language.

\begin{comment}
Programming by example~\cite{Lieberman2001} enables the user to write programs by
providing example input-output pairs. A related technique, programming by
demonstration~\cite{Cypher1993}, lets the user demonstrate individual steps
using a direct manipulation UI. Tools that embody these techniques often
generalize the user's actions to synthesize programs that operate on new,
unknown inputs~\cite{Yessenov2013, Kandel2011}.
% Common use cases include synthesizing text editing~\cite{Yessenov2013} and
% data cleaning~\cite{Kandel2011} scripts.
CodeInk takes inspiration from these tools and allows the user to explain
algorithms by demonstration, rather than by writing code. CodeInk does not yet
provide generalization capabilities; we plan to add that in future work.
\end{comment}
